\documentclass[10pt]{amsart}

\usepackage[T1]{fontenc}
\usepackage[margin=1in]{geometry}
\usepackage{enumitem}
\usepackage{mathpazo}
\usepackage{microtype}

\begin{document}
\title{Groupo\"ide fondemental et theoreme de van~Kampen en theorie de topos}
\author{Oliver Leroy}
\maketitle

\begin{center}
This work was suggested to me by C. Contou-Carrere and A. Grothendieck.

My results were demonstrated during algebraic geometry meetings with D. Alibert and C. Contou-Carrere.
\end{center}

\section*{Introduction}

The title is enough to delimit the subject; I put the essential explanations in the table of contents, thus
forming an
\begin{center}
ANALYTICAL TABLE.
\end{center}

\begin{enumerate}[label=\arabic*.]
  \item \textbf{Connected objects in a topos}~(p.~1)
    Brief expos\'e of the necessary notions to define a locally connected topos.
  \item \textbf{Locally constant objects and Galois objects}~(p.~5)
    \begin{enumerate}[label=2.\arabic*.]
      \item The locally constant objects of a topos correspond to the covers of a topological space or a scheme, regarded as sheaves.
      \item We prove for the locally constant objects of a locally connected topos the principal properties of covers of a locally connected space.
      \item The Galois objects correspond to Galois covers.
        The ``Galois theory'' class of locally constant objects trivialized by a Galois object in a connected topos.
        (In the \'etale topos of the spectrum of a field $k$, the Galois objects are the Galois extensions of $k$;
        we thus recover the classical Galois theory).
      \item Topos generated by the locally constant objects of a given locally connected topos $E$:
        the results of the next chapter will allow us to regard this topos, which consists of direct sums of locally contant objects of $E$, as the fundamental groupoid of $E$.
    \end{enumerate}
  \item \textbf{Locally Galois topoi and the fundamental groupoid}~(p.~23)
    The notion of a locally Galois topos will bind us to a tedious theory of ``pro-groupoids'';
    and it will allow us to define the fundamental groupoid of a topos by a universal property.
  \item \textbf{Inductive limits of topoi and van Kampen's theorem}~(p.~37)
    We define an inductive system of topoi using a fibred category in topoi over a category of indices.
    The Cartesian sections of this fibred category are the objects of the topos which is the inductive limit of the system.
    So the objects of an inductive limit of topoi appears as objects of the direct sum with some decent data.
    Theorem 4.5 is used to describe the fundamental groupoid of an inductive limit of locally connected topoi given their fundamental groupoids.
    The statement and proof of this theorem involve an auxilliary topos, sort of an intermediate gluing between the direct sum and the inductive limit, which is described in (4.3) and (4.4).
    I removed it in the corollary of Proposition~4.6.2, which directly describes the locally constant objects of the inductive limit.
    The advantage of the form (4.5) is to allow explicit calculations, which are developed in the points 4.6.3 and 4.6.7.
  \item[A.] \textbf{Appendix}: Categories fibred in topoi~(61).
  \item \textbf{Complements}
    \begin{enumerate}[label=5.\arabic*.]
      \item Fundamental groupoid of a locally connected topos at a point.
      \item Profinite fundamental groupoids.
    \end{enumerate}
\end{enumerate}

\begin{thebibliography}{4}
  \bibitem[1]{1} P.~Gabriel and P.~Zisman, {Calculus of fractions and homotopy theory}.
  \bibitem[2]{2} J.~Giraud, {Cohomologie non-ab\'elienne (pour les cat\'egories fibr\'ees)}.
  \bibitem[3]{3} A.~Grothendieck and J.~L.~Verdier, {Expos\'es I \`a IV du s\'eminaire de g\'eom\'etrie alg\'ebrique SGA~4}.
  \bibitem[4]{4} A.~Grothendieck, {Expos\'es V et IX du s\'eminaire SGA~1}.
\end{thebibliography}

\end{document}


